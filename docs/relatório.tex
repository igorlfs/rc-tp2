\documentclass[12pt]{article}

\usepackage[T1]{fontenc}        % Codificação para português (ao copiar e colar do documento, acentos são colados corretamente)
\usepackage[portuguese]{babel}  % Português (alguns pacotes usam termos da linguagem, por ex 'Referências')
\usepackage{hyperref}

\author{Igor Lacerda}
\title{Trabalho Prático II}
\date{Redes de Computadores}

\begin{document}

\maketitle

O objetivo do trabalho foi aprender sobre o gerenciamento de um servidor com múltiplos clientes, fazendo uso da linguagem C, por meio da implementação de serviço de \textit{blogs}. Esse objetivo foi atingido com sucesso.\footnote{Eu decidi não usar o bendito Arial 12 dessa vez.}

\section*{Desafios, Dificuldades e Imprevistos}

Tendo o trabalho anterior como inspiração, não foi tão difícil começar dessa vez (eu praticamente só copiei o antigo e fui removendo o que era específico dele). Mas havia uma certa dúvida se isso seria reaproveitável quando eu começasse a tornar o servidor \textit{multithreaded}. Acho que a especificação ficou um pouco ambígua em alguns detalhes: por exemplo, não foi claro se era necessário fazer um \textit{subscribe in} ou um \textit{subscribe to}: ambos aparecem nos exemplos da especificação. No final, esse exemplo foi resolvido no fórum como apenas \textit{subscribe}, mas é um sentimento angustiante: qualquer um desses detalhes pode descartar as horas de esforço investidas na implementação.

Acho que de maneira geral, essa angústia foi o maior desafio do trabalho. Depois que clicamos para enviar e o prazo acaba, não tem como voltar atrás. Sugiro que no futuro seja desenvolvido algum mecanismo para que nós, alunos, possamos validar nosso e não tenhamos que ficar preocupados com detalhes que realmente não deveriam importar. O trabalho já é complicado o suficiente por si só.

De fato, em quesitos técnicos, as principais dificuldades foram: o uso de múltiplas \textit{threads} e a função para publicar os \textit{posts}. Ambas são as dificuldades são parecidas e talvez estejam relacionadas com o fato de eu ainda estar fazendo Sistemas Operacionais. De qualquer maneira, o uso de múltiplas \textit{threads} no servidor foi resolvido consultando-se um especialista e sua \textit{playlist} do \href{https://www.youtube.com/watch?v=Pg_4Jz8ZIH4}{tema}. Os bons e velhos \textit{mutexes} vieram ao resgate, mas também foi usada uma \textit{pool} de \textit{threads} para limitar o tamanho do número de \textit{threads} que o programa criaria. Foram feitos mais alguns ajustes, seguindo-se a \textit{playlist}.

Para resolver a publicação de \textit{posts} foi necessária uma certa gambiarra: criar uma \textit{thread} no cliente só pra ouvir (e processar) as respostas do servidor, enquanto a \textit{thread} principal leria a entrada e a enviaria para o servidor. Enquanto isso, do lado do servidor, a função de publicação procura entre todos os clientes e envia para quem for necessário, usando a \textit{thread} associada a quem publicou. Realmente não sei se essa era a solução esperada, mas apesar de eu ter chamado de gambiarra, achei bem elegante.

Dessa vez, não tive nenhuma dificuldade relacionada ao uso da linguagem C. Claro que tive meus \textit{segfaults}, e senti falta de um \texttt{std::vector}, mas foi muito menos insuportável do que eu achei que seria. Achei um trabalho legal de fazer e bastante proveitoso, especialmente os aspectos não relacionados a redes em si, como o uso dos mecanismos para evitar condições de corrida.

\end{document}
